\documentclass[11pt]{article}\usepackage[]{graphicx}\usepackage[]{color}
%% maxwidth is the original width if it is less than linewidth
%% otherwise use linewidth (to make sure the graphics do not exceed the margin)
\makeatletter
\def\maxwidth{ %
  \ifdim\Gin@nat@width>\linewidth
    \linewidth
  \else
    \Gin@nat@width
  \fi
}
\makeatother

\definecolor{fgcolor}{rgb}{0, 0, 0}
\newcommand{\hlnum}[1]{\textcolor[rgb]{0,0,0}{#1}}%
\newcommand{\hlstr}[1]{\textcolor[rgb]{0,0,0}{#1}}%
\newcommand{\hlcom}[1]{\textcolor[rgb]{0.4,0.4,0.4}{\textit{#1}}}%
\newcommand{\hlopt}[1]{\textcolor[rgb]{0,0,0}{\textbf{#1}}}%
\newcommand{\hlstd}[1]{\textcolor[rgb]{0,0,0}{#1}}%
\newcommand{\hlkwa}[1]{\textcolor[rgb]{0,0,0}{\textbf{#1}}}%
\newcommand{\hlkwb}[1]{\textcolor[rgb]{0,0,0}{\textbf{#1}}}%
\newcommand{\hlkwc}[1]{\textcolor[rgb]{0,0,0}{\textbf{#1}}}%
\newcommand{\hlkwd}[1]{\textcolor[rgb]{0,0,0}{\textbf{#1}}}%

\usepackage{framed}
\makeatletter
\newenvironment{kframe}{%
 \def\at@end@of@kframe{}%
 \ifinner\ifhmode%
  \def\at@end@of@kframe{\end{minipage}}%
  \begin{minipage}{\columnwidth}%
 \fi\fi%
 \def\FrameCommand##1{\hskip\@totalleftmargin \hskip-\fboxsep
 \colorbox{shadecolor}{##1}\hskip-\fboxsep
     % There is no \\@totalrightmargin, so:
     \hskip-\linewidth \hskip-\@totalleftmargin \hskip\columnwidth}%
 \MakeFramed {\advance\hsize-\width
   \@totalleftmargin\z@ \linewidth\hsize
   \@setminipage}}%
 {\par\unskip\endMakeFramed%
 \at@end@of@kframe}
\makeatother

\definecolor{shadecolor}{rgb}{.97, .97, .97}
\definecolor{messagecolor}{rgb}{0, 0, 0}
\definecolor{warningcolor}{rgb}{1, 0, 1}
\definecolor{errorcolor}{rgb}{1, 0, 0}
\newenvironment{knitrout}{}{} % an empty environment to be redefined in TeX

\usepackage{alltt}
\usepackage{fullpage}
\usepackage{enumitem}
\setlist{parsep=5pt}

\usepackage{float}
\usepackage{amsmath}

\title{Stat 537 Assignment 5 (Individual Part)}
\author{Kenny Flagg}
\date{February 26, 2016}
\IfFileExists{upquote.sty}{\usepackage{upquote}}{}
\begin{document}

\maketitle



\begin{enumerate}

\setcounter{enumi}{5}
\item %6
My distance metric will be the number of miles between the cities when
traveling by water. Short overland sections are allowed (a la Lewis and Clark
carrying their boats across the Continental Divide). I know approximately where
the Missouri River and Continental Divide are, but otherwise I don't know much
about the hydrology of Montana so my distances are more-or-less based on
distance along the Missouri with an as-the-crow-flies component added on.

% latex table generated in R 3.2.3 by xtable 1.7-4 package
% Wed Feb 24 10:01:34 2016
\begin{table}[ht]
\centering
\begin{tabular}{rllllllllll}
  \hline
 & 1 & 2 & 3 & 4 & 5 & 6 & 7 & 8 & 9 & 10 \\ 
  \hline
Belgrade &  &  &  &  &  &  &  &  &  &  \\ 
  Billings &  &  &  &  &  &  &  &  &  &  \\ 
  Bozeman &  &  &  &  &  &  &  &  &  &  \\ 
  Butte &  &  &  &  &  &  &  &  &  &  \\ 
  Dillon &  &  &  &  &  &  &  &  &  &  \\ 
  Great Falls &  &  &  &  &  &  &  &  &  &  \\ 
  Helena &  &  &  &  &  &  &  &  &  &  \\ 
  Kalispell &  &  &  &  &  &  &  &  &  &  \\ 
  Lewiston &  &  &  &  &  &  &  &  &  &  \\ 
  Missoula &  &  &  &  &  &  &  &  &  &  \\ 
   \hline
\end{tabular}
\end{table}


\item %7

\item %8

\item %9

\item %10

\item %11

\end{enumerate}

\section*{R Code Appendix}

\begin{enumerate}

\setcounter{enumi}{5}
\item %6

\item %7

\item %8

\item %9

\item %10

\item %11

\end{enumerate}

\end{document}

